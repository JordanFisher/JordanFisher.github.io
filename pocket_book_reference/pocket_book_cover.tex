%% -*- mode: latex; mode: reftex; mode: flyspell; coding: utf-8; tex-command: "pdflatex.sh" -*-

\documentclass[11pt]{article}
\usepackage[paperheight=180.98mm,paperwidth=231.29mm,top=0.0cm,bottom=0.0cm,left=0.0cm,right=0.0cm]{geometry}
\usepackage[utf8]{inputenc}

\usepackage[utf8]{inputenc}
\usepackage[T1]{fontenc}
\usepackage[osf]{libertine}
%\usepackage{microtype}
\usepackage{tikz}

\let\ordinal\relax
\usepackage[us]{datetime}
\newdateformat{dotdate}{\THEYEAR.\twodigit{\THEMONTH}.\twodigit{\THEDAY}}

\def\construct{none}
%% \def\construct{red}

\begin{document}

\center

\vspace*{\stretch{1}}

\begin{tikzpicture}

\pgfmathsetlengthmacro{\width}{224.2mm}
\pgfmathsetlengthmacro{\height}{0.95\textheight}
%\pgfmathsetlengthmacro{\spinewidth}{8.74mm}
\pgfmathsetlengthmacro{\spinewidth}{10.74mm} % add 1mm for the fold

\pgfmathsetlengthmacro\x{(\width-\spinewidth)/2}
\pgfmathsetlengthmacro\mx{\width/2}
\pgfmathsetlengthmacro\sidewidthx{(\width-\spinewidth)/2}
\pgfmathsetlengthmacro\mxl{(\width-\spinewidth)/4}
\pgfmathsetlengthmacro\mxr{\width/2 + \spinewidth/2 + (\width-\spinewidth)/4}
\pgfmathsetlengthmacro\my{\height/2}

\draw[draw=\construct] (0,0) rectangle ++(\width,\height);
\draw[draw=\construct] (\x,0) rectangle ++(\spinewidth,\height);

\node[rotate=-90] (spine)
% at (\mx,\my)% proto 2023.06.10
 at (\mx-1mm,\my)
     {\large The Little Book of Things \hspace*{3cm} Bob Jones};

%%%%%%%%%%%%%%%%%%%%%%%%%%%%%%%%%%%%%%%%

\node (cover front) at (\mxr,\my){
%
  \begin{minipage}{80mm}
    \begin{center}

      \vspace*{18ex}

      {\huge The Little Book\\[0.75ex] of\\[1.75ex] Things}

      \vspace*{8ex}

      {\Large Bob Jones}

      \vspace*{12ex}

%%       \includegraphics[width=0.8\textwidth] {pics/neocognitron.png}\hspace*{3mm}

      \vspace*{18ex}

    \end{center}
  \end{minipage}
};

%%%%%%%%%%%%%%%%%%%%%%%%%%%%%%%%%%%%%%%%

\node (cover back)
% at (\mxl,\my) % proto 2023.06.10
at (\mxl-2mm,\my)
{

  \begin{minipage}{80mm}
This book is a short introduction to things for readers with a
STEM background. It aims at providing the necessary background to
understand landmark things.
  \end{minipage}
};

%%%%%%%%%%%%%%%%%%%%%%%%%%%%%%%%%%%%%%%%

\end{tikzpicture}

\vspace*{\stretch{1}}

\end{document}